\documentclass[a4paper,12pt]{article}

\usepackage{amsmath}
\usepackage{amsfonts}
\usepackage{amssymb}
\usepackage{amsthm}
\usepackage{amsrefs}
\usepackage[english]{babel}
\usepackage[all]{xy}
\usepackage{graphics,color}
\usepackage{ textcomp }
\usepackage{graphicx}

\setlength{\parindent}{0cm}
\setlength{\parskip}{0.5\baselineskip}


\newcommand{\HRule}{\rule{\linewidth}{0.5mm}}


\begin{document}

\begin{titlepage}

\begin{center}


\HRule \\[0.4cm]
{ \huge \bfseries Discussieforum}\\[0.4cm]
\HRule \\[0.4cm]

\vfill

\end{center}

\begin{minipage}{0.4\textwidth}
\begin{flushleft} \large
\emph{Gemaakt door:}\\
Michael Chen\\ Jenniger Roeting\\ Kyllian Broers \\ Max Bevelander \\
\emph{Groep: Webdb13KIC1}
\end{flushleft}

\end{minipage}
\vfill
{\large \today}

\end{titlepage}

\newpage
\begin{center}
{ \LARGE \bfseries Inhoudsopgave}\\[0.1cm]
\HRule \\[0.5cm]
\end{center}

\newpage
\begin{center}
{\LARGE \bfseries Deel 1 Informatie voor de klant}\\[0.1cm]
\HRule \\[0.5cm]
\end{center}

\newpage
\begin{center}
{ \LARGE \bfseries Inleiding}\\[0.1cm]
\HRule \\[0.5cm]
\end{center}
{\bfseries Opdracht}\\
De opdracht is het bouwen van een discussieforum voor een bedrijf, waardoor gebruikers eenvoudiger met elkaar contact kunnen houden. Zo kunnen klanten bijvoorbeeld vragen stellen over de producten van het bedrijf of feedback geven etc. Er moet een hi\"erarchie systeem in het forum zitten, zodat niet alle gebruikers dezelfde rechten hebben.
Voor dit forum zullen er drie soorten gebruikers zijn: beheerders, ingeschreven gebruikers en mensen die niet ingeschreven staan in het forum.\\

Beheerders\\
De mensen van het bedrijf zelf zijn  de beheerders van het forum. Zij moeten nieuwe fora kunnen maken en verwijderen. Het moet mogelijk zijn om per forum te kiezen of de ingezonden berichten direct in het forum geplaatst worden of eerst gescreend moeten worden door de beheerders.\\

Gebruikers\\
De gebruikers zijn klanten die zich voor  het forum geregistreerd hebben. Zij kunnen onder andere de goedgekeurde berichten lezen, zelf berichten plaatsen en reageren op andere berichten. \\

Niet-gebruikers\\
De niet-gebruikers kunnen alleen goedgekeurde berichten lezen. Het forum is voor de niet-gebruikers erg gelimiteerd. Om gebruiker te worden, kunnen zij zich registeren op het forum. \\



\newpage
\begin{center}
{ \LARGE \bfseries Functionele Eisen}\\[0.1cm]
\HRule \\[0.5cm]
\end{center}
Een discussieforum bestaat uit de volgende basis aspecten:\\\\
-	Homepagina;\\\\
-	Registreerpagina;\\\\
-	Inlogpagina;\\\\
-	Forum;\\\\
-	Ledenlijst;\\\\
-	FAQ-pagina\\\\




\newpage
\begin{center}
{ \LARGE \bfseries Ontwerp Beslissingen}\\[0.1cm]
\HRule \\[0.5cm]
\end{center}
{\bfseries Homepagina} \\
Content ontwerp van home.\\



{\bfseries Registreerpagina}\\
Content ontwerp van registreerpagina.\\

{\bfseries Inlogpagina}\\
Content ontwerp vaninlogpagina.\\


{\bfseries Forum}\\
Content ontwerp van forum.\\


{\bfseries Ledenlijst}\\
Content ontwerp van Ledenlijst.\\


{\bfseries FAQ-pagina}\\
Content ontwerp van FAQ-pagina.\\











\newpage
\begin{center}
{ \LARGE \bfseries Handleiding}\\[0.1cm]
\HRule \\[0.5cm]
\end{center}
een handleiding voor het gebruik van de site.


\newpage
\begin{center}
{ \LARGE \bfseries Reflectie}\\[0.1cm]
\HRule \\[0.5cm]
\end{center}
een reflectie op welke doelen bereikt zijn en welke niet.


\newpage
\begin{center}
{\LARGE \bfseries Deel 2 Informatie voor de ICT}\\[0.1cm]
\HRule \\[0.5cm]
\end{center}


\newpage
\begin{center}
{\LARGE \bfseries Software}\\[0.1cm]
\HRule \\[0.5cm]
\end{center}
de software die nodig is om de site te kunnen gebruiken.

\newpage
\begin{center}
{\LARGE \bfseries Installatieprocedure}\\[0.1cm]
\HRule \\[0.5cm]
\end{center}
de installatieprocedure voor eerste gebruik.

\newpage
\begin{center}
{\LARGE \bfseries Datamodel}\\[0.1cm]
\HRule \\[0.5cm]
\end{center}
het gebruikte datamodel (schema!)

\newpage
\begin{center}
{\LARGE \bfseries Structuur}\\[0.1cm]
\HRule \\[0.5cm]
\end{center}
de structuur van de PHP/javascript code

\newpage
\begin{center}
{\LARGE \bfseries Instructies}\\[0.1cm]
\HRule \\[0.5cm]
\end{center}
instructies met betrekking tot bv. onderhoud aan de site.
\end{document}
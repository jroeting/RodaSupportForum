\documentclass[a4paper,12pt]{article}

\usepackage{amsmath}
\usepackage{amsfonts}
\usepackage{amssymb}
\usepackage{amsthm}
\usepackage{amsrefs}
\usepackage[english]{babel}
\usepackage[all]{xy}
\usepackage{graphics,color}
\usepackage{ textcomp }
\usepackage{graphicx}

\setlength{\parindent}{0cm}
\setlength{\parskip}{0.5\baselineskip}


\newcommand{\HRule}{\rule{\linewidth}{0.5mm}}


\begin{document}

\begin{titlepage}

\begin{center}


\HRule \\[0.4cm]
{ \huge \bfseries Discussieforum}\\[0.4cm]
\HRule \\[0.4cm]

\vfill

\end{center}

\begin{minipage}{0.4\textwidth}
\begin{flushleft} \large
\emph{Gemaakt door:}\\
Michael Chen\\ Jenniger Roeting\\ Kyllian Broers \\ Max Bevelander \\
\emph{Groep: Webdb13KIC1}
\end{flushleft}

\end{minipage}
\vfill
{\large \today}

\end{titlepage}

\newpage
\begin{center}
{ \LARGE \bfseries Inhoudsopgave}\\[0.1cm]
\HRule \\[0.5cm]
\end{center}

\newpage
\begin{center}
{\LARGE \bfseries Deel 1 Informatie voor de klant}\\[0.1cm]
\HRule \\[0.5cm]
\end{center}

\newpage
\begin{center}
{ \LARGE \bfseries Inleiding}\\[0.1cm]
\HRule \\[0.5cm]
\end{center}
{\bfseries Opdracht}\\
De opdracht is het bouwen van een discussieforum voor een bedrijf, waardoor gebruikers eenvoudiger met elkaar contact kunnen houden. Zo kunnen klanten bijvoorbeeld vragen stellen over de producten van het bedrijf of feedback geven etc. Er moet een hi\"erarchie systeem in het forum zitten, zodat niet alle gebruikers dezelfde rechten hebben.
Voor dit forum zullen er drie soorten gebruikers zijn: beheerders, ingeschreven gebruikers en mensen die niet ingeschreven staan in het forum.\\

Beheerders\\
De mensen van het bedrijf zelf zijn  de beheerders van het forum. Zij moeten nieuwe fora kunnen maken en verwijderen. Het moet mogelijk zijn om per forum te kiezen of de ingezonden berichten direct in het forum geplaatst worden of eerst gescreend moeten worden door de beheerders.\\

Gebruikers\\
De gebruikers zijn klanten die zich voor  het forum geregistreerd hebben. Zij kunnen onder andere de goedgekeurde berichten lezen, zelf berichten plaatsen en reageren op andere berichten. \\

Niet-gebruikers\\
De niet-gebruikers kunnen alleen goedgekeurde berichten lezen. Het forum is voor de niet-gebruikers erg gelimiteerd. Om gebruiker te worden, kunnen zij zich registeren op het forum. \\



\newpage
\begin{center}
{ \LARGE \bfseries Functionele Eisen}\\[0.1cm]
\HRule \\[0.5cm]
\end{center}
Een discussieforum bestaat uit de volgende basis aspecten:\\\\
-	Homepagina;\\\\
-	Registreerpagina;\\\\
-	Inlogpagina;\\\\
-	Forum;\\\\
-	Ledenlijst;\\\\
- 	Profielpagina \\\\
-	FAQ-pagina;\\\\
-       Administrator panel\\\\




\newpage
\begin{center}
{ \LARGE \bfseries Ontwerp Beslissingen}\\[0.1cm]
\HRule \\[0.5cm]
\end{center}
{\bfseries Homepagina} \\
De homepagina bestaat uit een korte introductie over de inhoud van de website. Daarnaast worden op de homepagina de 10 meest recente en de 10 meest populaire berichten weergegeven. Hierdoor kan de gebruiker in \'e\'en oogopslag zien, welke onderwerpen nieuw zijn en uit welke onderwerpen de bezoeker mogelijk de meeste informatie kan halen. De onderwerpen die in deze lijsten verschijnen zijn klikbare links die verwijzen naar het berichtenoverzicht van het desbetreffende ondwerep.\\



{\bfseries Registreerpagina}\\
Content ontwerp van registreerpagina.\\

{\bfseries Inlogpagina}\\
Content ontwerp vaninlogpagina.\\


{\bfseries Forum}\\
De indeling van het forum is zodanig, dat het overzichtelijk is en intuitief in gebruik. Ook zonder handleiding moet een gebruiker in staat zijn om berichten te kunnen plaatsen en onderwerpen te kunnen vinden.
Het forum bestaat uit drie onderdelen. Een overzicht met de categorie\"en, een overzicht van de onderwerpen binnen een zekere categorie en een overzicht van de berichten in een zeker onderwerp. In het overzicht met categorie\"en worden categorie\"en weergegeven waarin de gebruiker onderwerpen kan plaatsen. Het indelen in categorie\"en maakt het forum overzichtelijk voor de gebruiker. De gebruiker kan zo zien waar deze moet zijn om een antwoord op zijn of haar vraag te vinden. Het overzicht met categorie\"en bestaat uit klikbare links, waarna de gebruiker terechtkomt in het onderwerpenoverzicht. Dit overzicht bestaat uit een lijst met alle onderwerpen die geplaatst zijn in een zekere categorie. Deze lijst bestaat uit klikbare links naar het berichtenoverzicht van een zeker onderwerp. Wanneer de gebruiker ingelogd is, heeft deze de mogelijkheid om een nieuw onderwerp te plaatsen door op de daarvoor bestemde knop te drukken. Voordat dit onderwerp zal verschijnen in het onderwerpenoverzicht, dient deze eerst goedgekeurd te worden door een administrator, zodat spamberichten of berichten in een verkeerde categorie voorkomen worden. Het berichtenoverzicht van een zeker onderwerp bestaat uit het beginbericht, waarmee het onderwerp geopend is en alle reacties van andere gebruikers. Indien een gebruiker ingelogd is, heeft deze de mogelijkheid om een reactie te plaatsen door zijn of haar reactie te schrijven, onderaan de pagina. Daarnaast heeft de gebruiker de mogelijkheid om de inhoud van zijn of haar eigen berichten te verwijderen en berichten van anderen als spam te rapporteren.\\


{\bfseries Ledenlijst}\\
Het forum is niet compleet zonder een ledenlijst. Hierin worden alle mensen die zich geregistreerd hebben als gebruiker weergegeven. Mensen die niet geregistreerd staan, kunnen de ledenlijst niet zien. Als een niet-gebruiker op de ledenlijst klikt, wordt diegene meteen doorverwezen naar de inlogpagina. Het alleen weergeven van de gebruikersnamen in de lijst was naar onze mening niet aantrekkelijk, dus hebben we besloten ook nog de registratiedata, aantal berichten die ze hebben geplaatst, de e-mailadressen en het account-type weer te geven. Als extra functie zijn de gebruikersnamen klikbare links. Als op een naam wordt klikt, wordt er doorverwezen naar zijn of haar profiel. \\

 
{\bfseries Profiel}\\
Ieder lid van het forum heeft een eigen profiel pagina ter beschikking, die hij/zij indien gewenst kan bewerken. Zo kan het avatar plaatje en de quote veranderd worden en kan de leeftijd, land en persoonlijkke tekst aan het profiel toegevoegd worden. Ook is er de mogelijkheid om het wachtwoord te veranderen. Dit is vooral handig als de gebruiker zijn of haar wachtwoord is vergeten. Als dat gebeurt, wordt er een mail gestuurd met een nieuw wachtwoord dat automatisch gegenereerd is. Als gebruiker zou het vervolgens wel fijn zijn om dat wachtwoord te kunnen wijzigen. Indien een gebruiker ook een administrator van het forum is, beschikt diegene ook een "adminitrator panel" in zijn of haar profiel. \\


{\bfseries Administrator panel}\\
Om het berichtenverkeer in het forum te handhaven, zijn er administrators toegewezen. De administrators kunnen enkel bepaald worden door de website beheerders. Administrators hebben hun eigen panel, deze is toegankelijk via hun eigen profiel. Hier vinden zij een overzicht van de andere administrators op het forum, een lijst met nog goed te keuren onderwerpen, een lijst met spamrapportages en een overzicht van nog niet geactiveerde accounts. Daarnaast hebben administrators de mogelijkheid om gebruikersaccounts te blokkeren wanneer zij voor overlast op het forum zorgen. De taken van een administrator zijn het blokkeren van gebruikers die de rust op het forum verstoren, het goed- of afkeuren van nieuwe onderwerpen, het nakijken van spamrapportages en het eventueel verwijderen van deze berichten, het verwijderen van onredelijke berichten en het verwijderen van gebruikersaccounts die nog niet geactiveerd zijn en langer dan twee weken terug aangemaakt zijn. \\


{\bfseries FAQ-pagina}\\
Om de niet-gebruikers en nieuwe gebruikers van het forum een handje te helpen, is er ook een FAQ-pagina aanwezig. Deze pagina is voor iedereen toegangkelijk. Daarin staan alle meest gestelde vragen met de bijbehorende antwoorden. De vragen zijn weergegeven als links en de antwoorden van die vragen komen te voorschijnen als er op geklikt wordt. Dus als gebruikers vragen hebben of problemen ondervinden, kunnen zij eerst de FAQ-pagina raadplegen. Om het forum netjes te houden, staan er ook een aantal regels vermeld in de FAQ-pagina.






\newpage
\begin{center}
{ \LARGE \bfseries Handleiding}\\[0.1cm]
\HRule \\[0.5cm]
\end{center}
een handleiding voor het gebruik van de site.


\newpage
\begin{center}
{ \LARGE \bfseries Reflectie}\\[0.1cm]
\HRule \\[0.5cm]
\end{center}
{\bfseries Doelen die bereikt zijn:}\\
Een forum moet minimaal de volgende aspecten bevatten:\\
-	Registreerpagina: 
Je moet als gebruiker kunnen registreren om het forum \hspace{30 cm}te kunnen gebruiken.\\
-	Inlogpagina:
Er moet een inlogpagina aanwezig zijn, zodat mensen als gebruikers kunnen inloggen.\\
-	Het forum met de geplaatste berichten:
De gebruikers moeten op het forum berichten kunnen plaatsen en op bestaande berichten reageren.\\
-	Home pagina:
De 10 meest recente berichten en 10 meest populaire berichten worden in de home pagina weergegeven.\\
-	Profielpagina:
Ieder lid heeft een eigen profiel.\\

Extra aspecten die toegevoegd zijn aan het forum:\\
-	Op de inlogpagina staat een extra link “password forgotten”. Als een gebruiker zijn of haar wachtwoord is vergeten, dan wordt er via de mail een nieuw wachtwoord verstuurd naar de gebruiker;\\
-	Op het forum is een ledenlijst aanwezig;\\
-	Regels die gebruikers op het forum moeten naleven en meest gestelde vragen in de FAQ-pagina;\\
-	Knopje om ongepaste berichten te melden;\\
-	Administratorpanel: wordt weergeven op profielpagina als diegene een administrator is;\\
-	Een gebruiker kan indien gewenst zijn of haar wachtwoord wijzigen.\\

{\bfseries Doelen die niet bereikt zijn:}\\
-	Zoekmachine;\\
-	Javascript gebruiken om velden te controleren waar gebruikers gegevens moeten invullen bijvoorbeeld bij het registreren, inloggen en het wijzigen van het profiel;\\
-	PM-systeem voor de leden;\\
-	Status van de leden: online of offline op het forum;\\

De bovenstaande doelen zijn niet bereikt en dat komt vooral wegens gebrek aan tijd.



\newpage
\begin{center}
{\LARGE \bfseries Deel 2 Informatie voor de ICT}\\[0.1cm]
\HRule \\[0.5cm]
\end{center}


\newpage
\begin{center}
{\LARGE \bfseries Software}\\[0.1cm]
\HRule \\[0.5cm]
\end{center}
De programmeeromgeving dient te bestaan uit een editor voor PHP en HTML, het programma dat hiervoor gebruikt kan worden is afhankelijk van de voorkeur van de beheerder. Relatief eenvoudige editors zoals Gedit zijn voldoende, maar ook kunnen programma's als Dreamweaver en Eclipse gebruikt worden. Daarnaast is er een SFTP of SSH file transfer benodigd. Hiervoor is WinSCP aanbevolen.

\newpage
\begin{center}
{\LARGE \bfseries Installatieprocedure}\\[0.1cm]
\HRule \\[0.5cm]
\end{center}
de installatieprocedure voor eerste gebruik.

\newpage
\begin{center}
{\LARGE \bfseries Datamodel}\\[0.1cm]
\HRule \\[0.5cm]
\end{center}
het gebruikte datamodel (schema!)
zie Relatie.pdf

\newpage
\begin{center}
{\LARGE \bfseries Structuur}\\[0.1cm]
\HRule \\[0.5cm]
\end{center}

Bijna alle pagina’s maken gebruik van de connectie met de database. Alleen de FAQ-pagina maakt geen gebruik ervan. Er wordt een connectie gemaakt met de database door het bestand 'db\_con.php'. In die database staan drie tabellen beschreven: \\
-	user\_data: alle gegevens van de geregistreerde gebruikers;\\
-	subjects: alle berichten die geplaatst zijn door de gebruikers van het forum;\\
-	posts: alle reacties van gebruikers op die berichten.

{\bfseries Registreren} \\
Het registeren gebeurt in het 'registerform.php' bestand. Daar worden de ingevulde velden (gegevens van de gebruiker) in de tabel 'user\_data' geplaatst. Natuurlijk worden de ingevulde velden eerst gecontroleerd. Na het invoeren van de gegevens wordt “register.php” aangeroepen. Daarin worden de velden gecontroleerd. Zo mag bijvoorbeeld een “username” geen spaties bevatten en het moet uniek zijn. Op deze manier worden alle velden gecontroleerd. 

{\bfseries Inloggen}\\
Tijdens het registeren zijn de velden, als alles goed is verlopen, opgeslagen in de tabel 'user\_data'. Om te kunnen inloggen, moet de gebruiker zijn of haar gebruikersnaam en wachtwoord invullen dat beschreven is door het bestand 'inlog.php'. Na het invoeren wordt de 'checklogin.php' aangeroepen. Daar wordt gekeken of de gebruikersnaam en het wachtwoord in de database staan en of het bij die gebruiker overeenkomen. Als alle stappen zijn verlopen zonder fouten, dan is de gebruiker ingelogd.

{\bfseries Ledenlijst}\\
De gebruikers die zich op het forum hebben geregistreerd en hun account hebben geactiveerd komen in de database te staan. Met behulp van de een query worden de gegevens per gebruiker opgehaald uit de database en vervolgens wordt daar een mooi lijst van gemaakt. 

{\bfseries Profiel}\\
De gegevens van de gebruiker worden getoond in het profiel. Indien gewenst kan het profiel en het wachtwoord van de gebruiker bewerkt worden en kunnen er zo nieuwe gegevens aan de database worden toegevoegd. Het 'editprofile.php' bestand is het formulier om de nieuwe gegevens in te vullen. Vervolgens worden deze velden gecontroleerd door het bestand 'profiel.php'. Voor het wachtwoord is er een apart bestand genaamd 'changepasswordform.php'. Eveneens wordt het invoer gecontroleerd door het bestand 'changepassword.php'. 

{\bfseries Forum}\\
Als een gebruiker een eigen bericht maakt of op andere berichten reageert, wordt dat opgeslagen in de tabellen subjects en posts. Met behulp van queries worden de berichten met de bijbehorende reacties getoond. Ook hier wordt al het invoer gecontroleerd voordat ze in de database worden ingevoerd.\\



\newpage
\begin{center}
{\LARGE \bfseries Instructies}\\[0.1cm]
\HRule \\[0.5cm]
\end{center}
instructies met betrekking tot bv. onderhoud aan de site.
\end{document}
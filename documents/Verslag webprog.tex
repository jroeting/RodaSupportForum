\documentclass[a4paper,12pt]{article}

\usepackage{amsmath}
\usepackage{amsfonts}
\usepackage{amssymb}
\usepackage{amsthm}
\usepackage{amsrefs}
\usepackage[english]{babel}
\usepackage[all]{xy}
\usepackage{graphics,color}
\usepackage{ textcomp }
\usepackage{graphicx}

\setlength{\parindent}{0cm}
\setlength{\parskip}{0.5\baselineskip}


\newcommand{\HRule}{\rule{\linewidth}{0.5mm}}


\begin{document}

\begin{titlepage}

\begin{center}


\HRule \\[0.4cm]
{ \huge \bfseries Discussieforum}\\[0.4cm]
\HRule \\[0.4cm]

\vfill

\end{center}

\begin{minipage}{0.4\textwidth}
\begin{flushleft} \large
\emph{Gemaakt door:}\\
Michael Chen\\ Jenniger Roeting\\ Kyllian Broers \\ Max Bevelander \\
\emph{Groep: Webdb13KIC1}
\end{flushleft}

\end{minipage}
\vfill
{\large \today}

\end{titlepage}

\newpage
\begin{center}
{ \LARGE \bfseries Inhoudsopgave}\\[0.1cm]
\HRule \\[0.5cm]
\end{center}

\newpage
\begin{center}
{\LARGE \bfseries Deel 1 Informatie voor de klant}\\[0.1cm]
\HRule \\[0.5cm]
\end{center}

\newpage
\begin{center}
{ \LARGE \bfseries Inleiding}\\[0.1cm]
\HRule \\[0.5cm]
\end{center}
{\bfseries Opdracht}\\
De opdracht is het bouwen van een discussieforum voor een bedrijf, waardoor gebruikers eenvoudiger met elkaar contact kunnen houden. Zo kunnen klanten bijvoorbeeld vragen stellen over de producten van het bedrijf of feedback geven etc. Er moet een hi\"erarchie systeem in het forum zitten, zodat niet alle gebruikers dezelfde rechten hebben.
Voor dit forum zullen er drie soorten gebruikers zijn: beheerders, ingeschreven gebruikers en mensen die niet ingeschreven staan in het forum.\\

Beheerders\\
De mensen van het bedrijf zelf zijn  de beheerders van het forum. Zij moeten nieuwe fora kunnen maken en verwijderen. Het moet mogelijk zijn om per forum te kiezen of de ingezonden berichten direct in het forum geplaatst worden of eerst gescreend moeten worden door de beheerders.\\

Gebruikers\\
De gebruikers zijn klanten die zich voor  het forum geregistreerd hebben. Zij kunnen onder andere de goedgekeurde berichten lezen, zelf berichten plaatsen en reageren op andere berichten. \\

Niet-gebruikers\\
De niet-gebruikers kunnen alleen goedgekeurde berichten lezen. Het forum is voor de niet-gebruikers erg gelimiteerd. Om gebruiker te worden, kunnen zij zich registeren op het forum. \\



\newpage
\begin{center}
{ \LARGE \bfseries Functionele Eisen}\\[0.1cm]
\HRule \\[0.5cm]
\end{center}
Een discussieforum bestaat uit de volgende basis aspecten:\\\\
-	Homepagina;\\\\
-	Registreerpagina;\\\\
-	Inlogpagina;\\\\
-	Forum;\\\\
-	Ledenlijst;\\\\
-	FAQ-pagina;\\\\
-         Administrator panel\\\\




\newpage
\begin{center}
{ \LARGE \bfseries Ontwerp Beslissingen}\\[0.1cm]
\HRule \\[0.5cm]
\end{center}
{\bfseries Homepagina} \\
De homepagina bestaat uit een korte introductie over de inhoud van de website. Daarnaast worden op de homepagina de 10 meest recente en de 10 meest populaire berichten weergegeven. Hierdoor kan de gebruiker in één oogopslag zien, welke onderwerpen nieuw zijn en uit welke onderwerpen de bezoeker mogelijk de meeste informatie kan halen. De onderwerpen die in deze lijsten verschijnen zijn klikbare links die verwijzen naar het berichtenoverzicht van het desbetreffende ondwerep.\\



{\bfseries Registreerpagina}\\
Content ontwerp van registreerpagina.\\

{\bfseries Inlogpagina}\\
Content ontwerp vaninlogpagina.\\


{\bfseries Forum}\\
De indeling van het forum is zodanig, dat het overzichtelijk is en intuitief in gebruik. Ook zonder handleiding moet een gebruiker in staat zijn om berichten te kunnen plaatsen en onderwerpen te kunnen vinden.
Het forum bestaat uit drie onderdelen. Een overzicht met de categorieen, een overzicht van de onderwerpen binnen een zekere categorie en een overzicht van de berichten in een zeker ondwerep. In het overzicht met categorieen worden categorieen weergegeven waarin de gebruiker onderwerpen kan plaatsen. Het indelen in categorieen maakt het forum overzichtelijk voor de bebruiker. De gebruiker kan zo zien waar deze moet zijn om een antwoord op zijn of haar vraag te vinden. Het overzicht met categorieen bestaat uit klikbare links, waarna de gebruiker terechtkomt in het onderwerpenoverzicht. Dit overzicht bestaat uit een lijst met alle onderwerpen die geplaatst zijn in een zekere categorie. Deze lijst bestaat uit klikbare links naar het berichtenoverzicht van een zeker onderwerp. Wanneer de gebruiker ingelogd is, heeft deze de mogelijkheid om een nieuw onderwerp te plaatsen door op de daarvoor bestemde knop te drukken. Voordat dit onderwerp zal verschijnen in het onderwerpenoverzicht, dient deze eerst goedgekeurd te worden door een administrator, zodat spamberichten of berichten in een verkeerde categorie voorkomen worden. Het berichtenoverzicht van een zeker onderwerp bestaat uit het beginbericht, waarmee het onderwerp geopend is en alle reacties van andere gebruikers. Indien een gebruiker ingelogd is, heeft deze de mogelijkheid om een reactie te plaatsen door zijn of haar reactie te schrijven, onderaan de pagina. Daarnaast heeft de gebruiker de mogelijkheid om de inhoud van zijn of haar eigen berichten te verwijderen en berichten van anderen als spam te rapporteren.\\


{\bfseries Ledenlijst}\\
Content ontwerp van Ledenlijst.\\


{\bfseries FAQ-pagina}\\
Content ontwerp van FAQ-pagina.\\


{\bfseries Administrator panel}\\
Om het berichtenverkeer in het forum te handhaven, zijn er administrators toegewezen. De administrators kunnen enkel bepaald worden door de website beheerders. Administrators hebben hun eigen panel, deze is toegankelijk via hun eigen profiel. Hier vinden zij een overzicht van de andere administrators op het forum, een lijst met nog goed te keuren onderwerpen, een lijst met spamrapportages en een overzicht van nog niet geactiveerde accounts. Daarnaast hebben administrators de mogelijkheid om gebruikersaccounts te blokkeren wanneer zij voor overlast op het forum zorgen. De taken van een administrator zijn het blokkeren van gebruikers die de rust op het forum verstoren, het goed- of afkeuren van nieuwe onderwerpen, het nakijken van spamrapportages en het eventueel verwijderen van deze berichten, het verwijderen van onredelijke berichten en het verwijderen van gebruikersaccounts die nog niet geactiveerd zijn en langer dan twee weken terug aangemaakt zijn.



\newpage
\begin{center}
{ \LARGE \bfseries Handleiding}\\[0.1cm]
\HRule \\[0.5cm]
\end{center}
een handleiding voor het gebruik van de site.


\newpage
\begin{center}
{ \LARGE \bfseries Reflectie}\\[0.1cm]
\HRule \\[0.5cm]
\end{center}
een reflectie op welke doelen bereikt zijn en welke niet.


\newpage
\begin{center}
{\LARGE \bfseries Deel 2 Informatie voor de ICT}\\[0.1cm]
\HRule \\[0.5cm]
\end{center}


\newpage
\begin{center}
{\LARGE \bfseries Software}\\[0.1cm]
\HRule \\[0.5cm]
\end{center}
De programmeeromgeving dient te bestaan uit een editor voor PHP en HTML, het programma dat hiervoor gebruikt kan worden is afhankelijk van de voorkeur van de beheerder. Relatief eenvoudige editors zoals Gedit zijn voldoende, maar ook kunnen programma's als Dreamweaver en Eclipse gebruikt worden. Daarnaast is er een SFTP of SSH file transfer benodigd. Hiervoor is WinSCP aanbevolen.

\newpage
\begin{center}
{\LARGE \bfseries Installatieprocedure}\\[0.1cm]
\HRule \\[0.5cm]
\end{center}
de installatieprocedure voor eerste gebruik.

\newpage
\begin{center}
{\LARGE \bfseries Datamodel}\\[0.1cm]
\HRule \\[0.5cm]
\end{center}
het gebruikte datamodel (schema!)
zie Relatie.pdf

\newpage
\begin{center}
{\LARGE \bfseries Structuur}\\[0.1cm]
\HRule \\[0.5cm]
\end{center}
de structuur van de PHP/javascript code

\newpage
\begin{center}
{\LARGE \bfseries Instructies}\\[0.1cm]
\HRule \\[0.5cm]
\end{center}
instructies met betrekking tot bv. onderhoud aan de site.
\end{document}